\documentclass[12pt,titlepage]{article}
\usepackage[margin=1.2in]{geometry}

\usepackage{amssymb}
\usepackage{amsmath}
\usepackage{amsthm}
\usepackage{graphicx}
\usepackage{hyperref}
% \hypersetup{
%     colorlinks=true,
%     linkcolor=blue,
%     filecolor=magenta,      
%     urlcolor=cyan,
%     pdftitle={Overleaf Example},
%     pdfpagemode=FullScreen,
%     }
\usepackage[dvipsnames]{xcolor}
\usepackage{array}
\usepackage{multirow}
\usepackage{tabularx}

\newcommand{\nat}{\mathbb{N}}
\newcommand{\real}{\mathbb{R}}
\newcommand{\complex}{\mathbb{C}}
\newcommand{\realn}[1]{\mathbb{R}^{#1}}
\newcommand{\infsum}[3]{\sum_{i=#1}^{#2} \left(#3\right)^{i}}
\newcommand{\forward}{\noindent ($\Longrightarrow$) \,\,}
\newcommand{\back}{\noindent ($\Longleftarrow$) \,\,}

\newcommand{\vecu}{\mathbf{u}}
\newcommand{\vecv}{\mathbf{v}}
\newcommand{\vecw}{\mathbf{w}}
\newcommand{\vecz}{\mathbf{0}}
\newcommand{\VNT}{V_{N,T}}
\newcommand{\DNT}{\mathcal{D}_{N,T}}
\newcommand{\FNT}{\mathcal{F}_{N,T}}
\newcommand{\FN}{\mathcal{F}_N}
\newcommand{\puretone}[2]{e^{\frac{2 \pi i #1}{#2}}}
\newcommand{\innerprod}[2]{\langle #1,#2 \rangle}

\newcommand{\indentt}{\hspace{.19in}}

\newenvironment{solution}{\noindent {\em Solution:} \\}{\hfill
\rule{1mm}{3mm}\bigskip}

\theoremstyle{definition}

\newtheorem{definition}{Definition}[subsection]
\newtheorem{question}[definition]{Question to Ponder}
\newtheorem{theorem}[definition]{Theorem}
\newtheorem{lemma}[definition]{Lemma}
\newtheorem{remark}[definition]{Remark}
\newtheorem{example}[definition]{Example}
\newtheorem{exercise}[definition]{Exercise}
\newtheorem{corollary}[definition]{Corollary}
\newtheorem{proposition}[definition]{Proposition}

\renewcommand{\qedsymbol}{\rule{1mm}{3mm}}
\begin{document}

\title{Musical Filters: The Linear Algebra Perspective}
\author{Evan Wagner}
\maketitle

\tableofcontents

\vspace{.2in}

$$***$$

\vspace{.2in}

\par \noindent CAUTION: This paper includes links to audio files. Some of the sounds are abrasive and can cause discomfort when played loudly. Make sure your audio is turned to a very low setting before you click a link, then adjust up if necessary. It's much better to not hear it the first time than to take a loud sawtooth wave point blank to the eardrum.

\par \bigskip \noindent If you are extra curious, here are links to the \href{https://drive.google.com/drive/folders/16oMcnn9lQVPj5wKAWLB81FRDqUVUpard?usp=sharing}{\color{blue} entire Drive folder of audio files} and the \href{https://docs.google.com/document/d/1_Np-1anndUBNAp1EMfoEQONGKTgCt1KJEKgOH2-CGKg/edit?usp=sharing}{\color{blue} MATLAB code} used to generate example audio and figures.

\par \bigskip \noindent I would like to thank Profs. Marie Snipes, Adam Lizzi, Judy Holdener, Bob Milnikel, Pamela Pyzza, and Carol Schumacher for their indispensable guidance throughout the capstone process.

\newpage

\input 1-Intro.tex

\input 2-Real-Fourier.tex

\input 3-Complex-Fourier.tex

\input 4-Analog-Filters.tex

\input 5-Discrete-Fourier.tex

\input 6-Digital-Filters.tex

\vspace{1in}

\begin{thebibliography}{99}
    \bibitem{Pierce} J. Pierce, ``Chapter 4: Sound Waves and Sine Waves." From {\em Music, Cognition, and Computerized Sound.} MIT Press, Cambridge, MA, 1999.
    \bibitem{Ryan} O. Ryan, {\em Linear Algebra, Signal Processing, and Wavelets -- A Unified Approach.} Springer, London, UK, 2019.
    \bibitem{Zhang} F. Zhang, {\em Matrix Theory.} Springer, New York, NY, 2011.
    \bibitem{Bretscher} O. Bretscher, {\em Linear Algebra with Applications.} Fifth ed., Pearson, Upper Saddle River, NJ, 2013.
\end{thebibliography}

\end{document}