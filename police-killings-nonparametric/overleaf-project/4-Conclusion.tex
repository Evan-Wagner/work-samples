
\section{Conclusion}

\par The nonparametric statistical techniques employed in this paper allow us to make several interesting conclusions about the recipients of deadly force from police in America in 2015. The reader is reminded not to extrapolate the following conclusions beyond the year 2015, and that our data includes the entire population of interest and thus its parameters are technically known. Our estimates of the parameters are not so useful in themselves, but our comparisons of the different populations do reveal whether or not differences between them are significant.

\par \bigskip We estimate that white victims are roughly 7 years older than nonwhite victims on average, and that the distribution of age is tighter around the median for nonwhite victims compared to white victims. We expand this conclusion with analysis of variance methods, determining with 95 percent overall confidence that white victims are typically 4-10 years older than Black victims and 5-12 years older than Latino victims, and that there is no meaningful difference in median age between Black and Latino victims. All our conclusions for this variable pair are very strong, so each nonparametric test came to the same conclusion as its parametric counterpart. However, the nonparametric conclusions are more useful because they are much more robust.

\par \bigskip We also estimate with 95 percent confidence that armed victims are between 0 and 6 years younger than unarmed victims on average. We find with a significance level of .1 that unarmed victims are between 0 and 9 years older than knife-wielding victims on average. On the other hand, we find no differences in the distributions of tract-level poverty rate based on whether or not the victim was armed, nor based on whether the victim specifically wielded a firearm, a knife, or no weapon. Nonparametric methods generally yield more conservative $p$-values in these analyses, but the Miller jackknife procedure comes closer to rejecting the null hypothesis of equal dispersion.

\par \bigskip Finally, we find with 95 percent confidence that, on average, white victims were killed in tracts with average personal incomes between \$608 and \$4623 higher than in the tracts of Black victims, and between \$186 and \$4816 higher than in the tracts of Latino victims. Distribution-free techniques prove considerably more confident than their counterparts for this pair of variables.

\par \bigskip Throughout our analyses, the nonparametric procedures proved to be resilient to both outliers and atypical distribution variances compared to their parametric cousins. Our data provided good examples of how transformations are unnecessary for rank-based nonparametric tests, giving them an advantage in this setting.

\newpage

\subsection{Further Discussion}

\par \bigskip Our conclusions point to other worthwhile investigations using these data. One interesting thread to pull would be how closely the victim population represents the general population. For instance, to look for significant racial disparities, one could group victims geographically and compare the racial proportions to the general population of each area using variables like $share\_white$, $share\_black$, and $share\_hispanic$. There is certainly no way to design an ethical experiment about police use of deadly force, so future analysis would have to remain in the realm of relations without making inferences about causations.