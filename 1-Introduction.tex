
\section{Introduction}

\par In recent years, a spate of high-profile violent encounters with American police officers has focused public attention on such incidents. Media outlets have responded by compiling detailed data on fatal incidents across the U.S. and making it available for statistical analysis. For one, \textit{FiveThirtyEight} compiled data on all such incidents from the year 2015, pulling details of each incident from data gathered by \textit{The Guardian} and adding location-related data from the American Community Survey (ACS) conducted in 2015 by the U.S. Census.\cite{dset} We will attempt to derive statistically significant conclusions from this dataset.

\par \bigskip As shall be seen in the following, the nature of these data prevents us from applying many techniques typically used for statistical analysis. In short, we cannot assume that the populations underlying the parameters we wish to measure are normally distributed, which is a prerequisite for all of the basic techniques taught in beginning statistics courses. Luckily, several savvy statisticians have developed statistical techniques whose conclusions are valid without assuming its parameters follow a distribution, normal or otherwise. We will apply their distribution-free methods to our data on police killings to try and answer several questions of interest.

\par \bigskip Before we begin, two very important hesitations must be addressed. The first is that our conclusions only apply to fatal encounters in the year 2015. With only one year in our sample, we cannot assess the influence of time-based effects on our parameters of interest. Indeed, as police brutality began growing more prominent in the public imagination starting in 2014, it is likely that some police departments have responded to scrutiny of their peacekeeping practices by overhauling them.

\par \bigskip Secondly, our data is not a typical simple random sample (SRS) because it technically includes our entire population of interest: 2015 victims of deadly force. As such, we have no need to estimate means and deviations because we can calculate them directly. That said, our conclusions are still meaningful for comprehending and comparing these data. If a higher power wound back the clock to 2015 and set the conditions back to how they were, we would expect slightly different outcomes due to the contribution of randomness to our parameters. 

\subsection{Data Preparation}

\par In order to apply our techniques, changes were made to the raw data. Rows which contained at least one missing value of a quantitative variable were removed, as were entries labeled \textit{``Unknown"} for some categorical variable; this reduced the number of observations from 467 to 438. Additionally, three new binary variables were created by forming two groups from the levels of categorical variables. The full list of variables is displayed in Table \ref{tab:variables}.

\begin{table}[h]
    \centering
    \begin{tabular}{|l|l|l|l|}
        \hline
        \textbf{Variable} & \textbf{Type} & \textbf{Description} & \textbf{Original Source}\\
        \hline
        $age$ & quantitative & Age of deceased & \textit{The Guardian}\\
        &&&\\
        \hline
        $pop$ & quantitative & Population of tract on which the & U.S. Census\\
        && killing occurred. Tracts are &\\
        && subdivisions of counties containing &\\ && a few thousand residents each. &\\
        \hline
        $share\_white$ & quantitative & Share of tract population that is & U.S. Census\\
        && non-Hispanic white &\\
        \hline
        $share\_black$ & quantitative & Share of tract population that is & U.S. Census\\
        && black (alone, not in combination) &\\
        \hline
        $share\_hispanic$ & quantitative & Share of tract population that is & U.S. Census\\
        && Hispanic/Latino (any race) &\\
        \hline
        $p\_income$ & quantitative & Tract-level median personal income & U.S. Census\\
        && (USD) &\\
        \hline
        $h\_income$ & quantitative & Tract-level median household & U.S. Census\\
        && income (USD) &\\
        \hline
        $county\_income$ & quantitative & County-level median household & U.S. Census\\
        && income (USD) &\\
        \hline
        $comp\_income$ & quantitative & Ratio of $h\_income$ to $county\_income$ & Calculated by\\
        &&& \textit{FiveThirtyEight}\\
        \hline
        $pov$ & quantitative & Tract-level poverty rate (official) & U.S. Census\\
        &&&\\
        \hline
        $urate$ & quantitative & Tract-level unemployment rate & Calculated by\\
        &&& \textit{FiveThirtyEight}\\
        \hline
        $college$ & quantitative & Share of tract population $\ge$ 25 years & Calculated by\\
        && old with BA or higher & \textit{FiveThirtyEight}\\
        \hline
        \hline
        $gender$ & categorical & Gender of deceased & \textit{The Guardian}\\
        \hline
        $raceethnicity$ & categorical & Race/ethnicity of deceased & \textit{The Guardian}\\
        \hline
        $cause$ & categorical & Cause of death & \textit{The Guardian}\\     \hline
        $armed$ & categorical & How/whether deceased was armed & \textit{The Guardian}\\
        \hline
        $county\_bucket$ & categorical & Tract-level household income, & Calculated by\\
        && quintile within county & \textit{FiveThirtyEight}\\
        \hline
        $nat\_bucket$ & categorical & Tract-level household income, & Calculated by\\
        && quintile nationally & \textit{FiveThirtyEight}\\
        \hline
        \hline
        $white$ & binary & Coded 1 when $raceethnicity =$ & Computed by\\
        && $``White"$, 0 otherwise & the author\\
        \hline
        $male$ & binary & Coded 1 when $gender = ``Male"$, & Computed by\\
        && 0 when $gender = ``Female"$ & the author\\
        \hline
        $armedyes$ & binary & Coded 0 when $armed = ``No"$, & Computed by\\
        && 1 otherwise & the author\\
        \hline
    \end{tabular}
    \caption{Variables in the dataset}
    \label{tab:variables}
\end{table}